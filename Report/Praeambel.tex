\documentclass[a4paper,10pt]{scrartcl}

\usepackage{units}
\usepackage{mathtools}
%\usepackage{a4wide}
\usepackage{bbm}
\usepackage{amssymb}
\usepackage{amsthm}
\usepackage{amsmath}
\usepackage{graphicx, enumerate}
\usepackage{psfrag}
\usepackage[matrix, arrow, curve]{xy}
\usepackage{array,tabularx}
%\usepackage[square, comma, numbers]{natbib}
\usepackage[unicode, a4paper]{hyperref}
\usepackage{subfigure}
\usepackage{algorithm2e}
\usepackage{url}
\usepackage{mathtools}
\usepackage{pgfplots}
\usepackage{siunitx}

\newcounter{myFCounter}[section]
\newcommand{\myFigure}[3]{%
    \begin{center}\begin{minipage}[t]{\columnwidth}%
    \begin{center}\refstepcounter{myFCounter}\vspace{1ex}%
    \includegraphics[width=#1\columnwidth,keepaspectratio]{#2}\ \\%
     Abb. \arabic{myFCounter}:\ \rm #3 
    \vspace{1ex}\end{center}%
    \end{minipage}\end{center}}
    \newcommand\myColumnSep[1]{%
\setlength{\columnsep}{#1}%
}

\newcommand{\inner}[2]{\left\langle #1, #2 \right\rangle}
\newcommand{\abs}[1]{\left| #1 \right|}

%die folgenden Zeilen sind für die Kopf und Fusszeile
%die eckigen Klammern betreffen die Titelseite, die geschwungenen Klammern das restliche Dokument 
\usepackage[automark]{scrpage2}
\pagestyle{scrheadings}
\ihead{\includegraphics[width=0.25 \textwidth]{figures/NTB-FHO_LOGO_2}} % linke Kopfzeile 
\chead{Estimating Outdoor Signal Strength of LoRaWAN} % mittlere Kopfzeile
\ohead{\includegraphics[width=0.15 \textwidth]{figures/logo_thingslogic}} % rechte Kopfzeile
%\cfoot[...]{\thepage} % mittlere Fusszeile 
\ofoot[...]{\today} % rechte Fusszeile 
\ifoot[...]{} 
\setheadsepline{0.5pt}
\setfootsepline{0.5pt}
%Befehle für kopf und fusszeile fertig

\setlength{\parindent}{0pt}
\setlength{\footskip}{1cm}

\linespread{1.5} %Für Zeilenabstand 1.5 muss linespreadfaktor auf 1.25 eingestellt werden.
%\onehalfspacing%Zeilenabstand (alternativ zu linespread

% Zeilenabstand verringern bei itemize
\let\origitemize\itemize
\def\itemize{\origitemize\itemsep0pt}
\setlength{\topmargin}{-1.5cm}  		%	Abstand Seitenkopf vom oberen Rand plus 1 Inch
\setlength{\textwidth}{17cm} 				%	{6.25in}  %Breite des Seitenrumpfes (15.8cm)
\setlength{\oddsidemargin}{-0.5cm}	%	{0in} %Abstand Text vom linken Rand plus ein Inch (=2.54cm)
\setlength{\textheight}{9.0in}  		%	Texthöhe ohne Kopfzeile und Fusszeile (23.9cm)
\setlength{\headheight}{1.3cm}			%	Kopfzeilenhöhe
\setlength{\headsep}{0.5cm}				%	Abstand Kopfzeile - Text
%\setlength{\headsep}{1cm}						%	Abstand Kopfzeile - Text
\setlength{\columnsep}{.5cm}				%	Abstand der 2 Kolonnen
%
\pagestyle{scrheadings}							%	Einzug bei Absatzbeginn unterdrücken

%\renewcommand{\figurename}{Abb.}		%	Text zur Bildbeschriftung
%\renewcommand{\tablename}{Tab.}			%	Text zur Tabellenbeschriftung
%\renewcommand{\refname}{Literaturverzeichnis}
%\let\mathmu\mu
%\def\mu{\ensuremath{\mathmu}}

\usepackage{tikz}
\usetikzlibrary{calc}
\usetikzlibrary{patterns}
%\usetikzlibrary{angles,quotes}



% theoremstyles 
\theoremstyle{plain}

\newtheorem{thm}{Theorem}[section]
\newtheorem{prop}[thm]{Proposition}
\newtheorem{ass}[thm]{Assumption}

\newtheorem{assa}{Assumption}
\renewcommand{\theassa}{\Alph{assa}}

\theoremstyle{definition}
\newtheorem{rem}[thm]{Remark}
\newtheorem{alg}[thm]{Algorithm}
\newtheorem{lem}[thm]{Lemma}
\newtheorem{dfn}[thm]{Definition}
\newtheorem{cor}[thm]{Corollary}
\newtheorem{que}{Question}
\newtheorem{example}[thm]{Example}
\newtheorem*{example*}{Example}

\newtheorem*{dfn*}{Definition}
\newtheorem*{alg*}{Algorithm}



\theoremstyle{remark}
\newtheorem{prob}{Problem}

